\documentclass{article}
\usepackage[utf8]{inputenc}
\usepackage{babel}
\usepackage{csquotes}
\usepackage{hyperref}
\usepackage{biblatex}
\addbibresource{NeuromorphicSensor.bib}

\title{Bibliographie capteurs neuromorphiques}
\author{Saulquin Aurélie}
\date{Octobre 2021}


\begin{document}

\maketitle

\section{Introduction}

Le calcul neuromorphique s'est énormément développé durant les dernières années. En effet, il répond à un besoin majeur de l'informatique : intégrer des intelligences artificielles aux systèmes embarqués. Les méthodes classiques de l'intelligence artificielle (IA) sont très efficaces mais également très coûteuses en énergie. L'avantage des systèmes neuromorphiques est leur faible consommation d'énergie. Pour interagir avec l'environnement, les chercheurs et les ingénieurs ont développé des capteurs bio-inspirés. Ces capteurs consomment également très peu d'énergie. D'une part car ils sont conçus ainsi, d'autre part car les capteurs neuromorphiques n'envoient pas ou très peu de données redondantes, réduisant ainsi la consommation due à la communication \cite{liu_neuromorphic_2010} \cite{vanarse_review_2016}. De plus, il est plus efficace d'envoyer des données dans un format directement compréhensible par le calculateur \cite{liu_neuromorphic_2010}. C'est pour toutes ces raisons que les capteurs neuromorphiques se développent rapidement et prennent de plus en plus d'importances dans les systèmes embarqués. Dans cette bibliographie, nous allons entrevoir une partie des capteurs bio-inspirés neuromorphiques. Nous allons notamment nous concentrer sur les capteurs de vision, les capteurs audio et les capteurs olfactif, même s'il existe d'autres capteurs neuromophiques.

\newpage
\section{Capteur de Vision}
Les capteurs de vision classiques sont basés sur l'envoi d'une image complète vers le calculateur, et cela plusieurs fois par seconde. Cette méthode induit une énorme quantité de données à traiter mais également des données souvent redondantes. L'envoi de données étant énergivore, l'envoi de données redondantes rend le capteur difficilement intégrable dans des systèmes embarqués à basse consommation d'énergie.\\ 
Un capteur de vision à basse consommation d'énergie peut être basé sur la rétine biologique. C'est d'ailleurs la première approche \cite{vanarse_review_2016} qui a été proposée, avec la rétine en silicone en 1994\cite{mahowald_silicon_1994}.
Aujourd'hui, deux grandes familles de caméras neuromorphiques dominent le marché, chacune basée sur une même technologie.\\
Le premier modèle de caméra est la caméra \textbf{DVS} \textit{Dynamic Vision Sensor} basée sur la technologie du pixel DVS \cite{lichtsteiner_128_2006}. Le pixel DVS va détecter un changement de luminosité relatif d'un certain pourcentage. Lorsque le changement de luminosité devient supérieur à un seuil fixé lors de la conception de la caméra, le pixel envoie une impulsion avec le protocole AER \textit{(Adress Event Representation)} pour aller exciter le neurone qui traitera l'information. Cette méthode permet à chaque pixel d'envoyer des impulsions de manière asynchrone, mais surtout l'envoi d'une impulsion correspond forcement à un changement d'information vidéo. Ainsi, seulement des informations importantes sont transmises et aucune donnée redondante n'est envoyée au calculateur. Après l'apparition de cette caméra, beaucoup de travaux \cite{delbruck_temporal_2010} ont eu pour objectif de baisser le seuil d'envoi d'impulsion pour récolter des informations de plus en plus précises.\\
Les caméras DVS sont efficaces et consomment très peu d'énergie, cependant l'information n'est traitable que par des réseaux neuromorphiques. Un autre problème des caméra DVS est que si l'image est fixe, ou si une partie de l'image ne change jamais, il peut être compliqué de détecter certaines formes. C'est pour tenter de résoudre ce problème que des études ont cherché a envoyer des informations supplémentaires que simplement sur le changement d'intensité lumineuse. Une étude \cite{posch_qvga_2011} a développé une caméra qui envoie des nuances de gris tout en consommant le moins d'énergie.\\
La caméra qui s'est imposée comme un bon compromis entre la DVS et la caméra classique est la caméra \textbf{DAVIS} \textit{apsDVS}. Cette caméra intègre un pixel composé de deux parties, une partie DVS qui détecte le changement d'intensité lumineuse, et un pixel APS \textit{(Active-Pixel Sensor)} qui va capter la lumière en nuances de gris \cite{brandli_240_2014}. La caméra DAVIS possède donc deux sorties, une sortie AER basée sur l'information des pixels DVS, et une autre sortie plus classique utilisable par les algorithmes de traitement vidéo habituels. \\
Nous avons donc vu que deux grandes familles de caméras neuromorphiques dominent le marché. Du point de vue de la consommation énergétique, la caméra DVS consomme 30 mW \cite{lichtsteiner_128_2006} pour la première version, mais certaines études arrivent à réduire la consommation à 4mW. \cite{serrano-gotarredona_128x128_nodate}. La caméra DAVIS consomme un peu plus que les caméra DVS récentes, entre 5mW et 14mW \cite{brandli_240_2014}. L'énergie consommée par les capteurs va également dépendre de l'activité impulsionnelle. 

\newpage
\section{Capteur auditif}
Un autre sens que les chercheurs ont voulu développer, c'est le sens de l'audition. Les capteurs neuromorphiques sont largement inspirés de la cochlée biologique.\\
L'un des premiers articles de capteur de son bio-inspiré \cite{lyon_analog_1988} a posé les bases du principe du capteur de son bio-inspiré. Le capteur est composé d'un ensemble de filtres passe-bas afin de sélectionner les différentes fréquences du signal sonore et ainsi pouvoir étudier, via un système électronique, la composition du son. Le capteur a été amélioré en améliorant la qualité des filtres, et surtout en proposant une gamme d'intensité sonore dynamique permettant de réduire les sons ou les fréquences d'intensité trop forte \cite{watts_improved_1992}. Enfin, la dernière grande amélioration des capteurs sonores bio-inspirés est le développement des structure en 2D \cite{hamilton_active_2008}. Cette structure permet de mieux simuler le comportement du fluide de la cochlée biologique, permettant d'ajouter l'effet de non linéarité. En effet, la cochlée humaine est moins sensible à certaines fréquences et la structure en 2D du capteur a permis de rendre le capteur moins sensible à certaines fréquences.\\
Il reste encore la question de la production et l'envoi des impulsions correspondant aux données récupérées par le capteur \cite{chan_aer_2007}. A la sortie de chaque filtre, on obtient une courbe qui se rapproche d'une sinusoïdale d'une certaine amplitude et d'une certaine fréquence. On mesure l'amplitude entre deux crêtes de cette courbe. Le résultat est une tension qui sera convertie par une système électronique en courant électrique. Ce courant est envoyé à un neurone de type integrate-and-fire qui produira une impulsion électrique qui sera envoyer sous format AER.\\
Grâce à cette méthode, il est possible d'effectuer les traitements classiques de l'IA sur les signaux sonores tels que la reconnaissance de locuteur ou la déduction de mot, la combinaison de deux cochlées artificiel permettant également de localiser la provenance du son \cite{liu_neuromorphic_2010}. En effet les impulsions reçues émises par le capteur sont, en plus d'une représentation fréquentielle du signal, une représentation temporelle du signal sonore. L'impulsion est émise lorsque le signal est reçu à un instant t. Si on place deux capteurs espacés d'un certain espace, on va bien recevoir les mêmes impulsions étant donné que c'est issu du même signal, cependant les impulsions émises par les capteurs seront décalées dans le temps. Ce décalage peut permettre de rendre plus simple la localisation de l'émetteur du signal sonore. D'autres études utilisent des architectures différentes, qui ne sont plus constituées intégralement de CMOS mais également de matériaux comme les MEMS \textit{(Micro Electronic Mechanical System)} couplés à un algorithme bio-inspiré pour la localisation de l'émetteur \cite{van_schaik_neuromorphic_2004}.\\
D'un point de vue énergétique, la technologie MEMS semble être moins consommatrice, l'article \cite{van_schaik_neuromorphic_2004} propose un capteur MEMS qui consomme 2 mW. Les technologies plus classiques à base de CMOS proposent un capteur qui consomme 50mW environ \cite{hamilton_active_2008}.

\newpage
\section{Capteur olfactif}
Plus qu'un simple capteur d'odeurs comme le nez humain, la famille des capteurs olfactifs regroupe en fait les capteurs de toutes sortes de substances chimiques sous forme de gaz. De manière générale, les capteurs de gaz classiques sont souvent calibrés pour détecter des gaz précis. D'autres manières de détecter des gaz ont été développées, comme la chromatographie ou encore la spectroscopie infrarouge à transformée de Fourier \cite{chiu_towards_2013}. Cependant ces méthodes sont soit difficilement portables, soit conçues pour détecter des gaz précis. Les capteurs bio-inspirés sont quant à eux plus portatifs et permettent de détecter n'importe quel gaz plus rapidement \cite{chiu_towards_2013}.\\
Le premier nez artificiel bio-inspiré s'inspire de la membrane absorbante du nez humain, qui absorbe les molécules odorantes pour les détecter. L'idée proposée par l'article \cite{noauthor_instrument_nodate} est de simuler cette membrane à l'aide d'un gel composé de plusieurs matériaux semi-conducteurs.\\
Avec le développement de la technologie CMOS \textit{(Complementary Metal Oxide Semi-Conductor)}, ce n'est plus cette méthodes qui est utilisée mais d'autres, basées à la fois sur la technologie CMOS, l'utilisation de MEMS mais également sur les MOX \textit{(Metal OXide)} pouvant offrir un plus large spectre de gaz détectables \cite{ng_cmos_2011} \cite{gardner_cmos_2010}.\\
L'intérêt du calcul neuromorphique pour la reconnaissance des gaz réside dans le fait que le capteur va envoyer un pattern temporel d'impulsions permettant ainsi d'utiliser la méthode de reconnaissance de pattern \cite{chiu_towards_2013}. Cette méthode permet de reconnaître des compositions chimiques complexes \cite{ng_cmos_2011} \cite{koickal_analog_2007}. La méthode pour convertir les données analogiques produites par les capteurs diffère, certains utilisent le modèle de ORN \cite{koickal_analog_2007}, d'autres utilisent la structure complète de la partie sensitive du capteur pour détecter certains paramètres pouvant être facilement convertis en impulsions avec un codage temporel \cite{ng_cmos_2011}.

\newpage
\section{Conlusion}
Dans cette bibliographie, nous n'avons parlé que de trois types de capteurs particuliers.Nous avons d'abord étudié les capteurs de vision, avec notamment les caméras DVS et DAVIS. Puis nous avons les capteurs de son et leur intérêt pour la localisation de l'émetteur. Enfin nous avons parlé des capteurs olfactifs et de manière plus général les capteurs de gaz neuromorphiques. Il faut cependant garder à l'esprit que d'autres capteurs ont été développés pour être intégrer dans des systèmes neuromorphiques. Par exemple des capteurs de toucher et de pression neuromorphiques \cite{rongala_neuromorphic_2017}.\\
On remarque que l'origine de la technologie ne vient pas d'une modification des capteurs déjà existants mais s'inspire beaucoup de la biologie. C'est avec le temps que le capteur se perfectionne avec des technologies électroniques modernes. Cette méthode permet d'obtenir des capteurs qui consomment très peu d'énergie et qui sont compatibles avec les calculateurs neuromorphiques, eux aussi très peu consommateurs. L'union des deux est donc une solution viable pour intégrer des intelligences artificielles dans des systèmes embarqués à faible consommation d'énergie. \\
Ces améliorations ont été permises par une meilleur compréhension des mécanismes biologiques qui régissent les sens et l'envoi des informations au cerveau, mais également par le développement de méthodes neuromorphiques comme le protocole AER, largement utilisé par les capteurs.\\
Les futurs recherches devront se concentrer sur le développement de réseaux de capteurs neuromorphiques \cite{liu_neuromorphic_2010} incluant plusieurs types de capteurs. L'augmentation des capteurs fait engendrer une augmentation du nombre d'impulsions qui vont devoir être gérées par le calculateur, ce qui peut poser problème à cause de la sur-abondance d'informations à traiter.

\medskip

\printbibliography 

\end{document}

